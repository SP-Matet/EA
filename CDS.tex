% Tout ce qui est mis derrière un « % » n'est pas vu par LaTeX
% On appelle cela des « commentaires ».Les commentaires permettent de
% commenter son document - comme ce que je suis en train de faire
% actuellement - et de cacher du code - cf. la ligne \pagestyle.


\documentclass[a4paper]{article}
\usepackage[utf8]{inputenc}
\usepackage{amsmath}
\usepackage{amsfonts}
\usepackage{amssymb}
\usepackage{amsthm}
\usepackage{graphicx}

\usepackage[a4paper]{geometry}% Réduire les marges
% \pagestyle{headings}        % Pour mettre des entêtes avec les titres
                              % des sections en haut de page

\title{Credit Valuation Adjustment sur Credit Default Swap}           % Les paramètres du titre : titre, auteur, date
\author{Simon Matet \and Antoine Sauvage}
\date{}                       % La date n'est pas requise (la date du
                              % jour de compilation est utilisée en son
			      % absence)

\sloppy                       % Ne pas faire déborder les lignes dans la marge

\begin{document}

\maketitle                    % Faire un titre utilisant les données
                              % passées à \title, \author et \date

\begin{abstract}
Le Credit Default Swap (CDS) est un type de contrat permettant à un créancier de s'assurer contre le risque de défaut d'un débiteur.
 
Cependant, l'institution émettrice d'un CDS peut elle même faire défaut avant ou en même temps que le débiteur.
 La prise en compte de cette possibilité dans le \textit{pricing} d'un CDS et les stratégies pour se prémunir de ce risque de contrepartie font l'objet de ce rapport.

\end{abstract}

\tableofcontents              % Table des matières

% \part{Titre}                % Commencer une partie.
.
.


\section{Présentation générale}               % Commencer une section, etc.


\subsection{Contexte et notations}         % Section plus petite
Nous appelerons un CDS sans risque un CDS tel que la probabilité de défaut de la contrepartie est nulle et un CDS avec risque un CDS où cette probabilité est strictement positive.
 Dans le cadre de ce rapport, une firme 0 détient une unité de dette à risque de la firme 1, qui peut faire défaut.
 Dans ce cas elle rembourse ses dettes à hauteur de $R_{1} \in \left[ 0, 1 \right[$.
 Pour se couvrir contre ce risque, la firme 0 achète un CDS à la firme 2 et s'engage à verser une prime $\kappa$ en échange d'une assurance contre le défaut de 1 jusqu'à la date de maturité $T$.
 La valeur du contrat dépend cependant de la probabilité que l'assureur 2 fasse défaut avant son terme.
\\ \\
Nous noterons donc :\\
$\beta (t)$ Le taux d'actualisation\\
$T$ La date de maturité du contrat\\
$R_{1}$ Le taux de recouvrement de la firme 1\\
$R_{2}$ Le taux de recouvrement de la firme 2\\
$\tau_{1} (\omega)$ Temps aléatoire du défaut de la firme 1\\
$\tau_{2} (\omega)$ Temps aléatoire du défaut de la firme 2\\
$\kappa$ La prime versée par le détenteur du CDS à la firme 2 à chaque unité de temps\\
$p (t, \omega)$ Le cash flow à venir après la date $t$ pour une réalisation $\omega$ de la firme 2 vers la firme détentrice du CDS en supposant que 2 ne fasse jamais défaut\\
$\pi (t, \omega)$ Le cash flow à venir après la date $t$ pour une réalisation $\omega$ de la firme 2 vers la firme détentrice du CDS dans le cas où 2 peut faire défaut\\
$P (t, \omega) = \mathbb{E}_{t} [p]$ le cash-flow espéré conditionnellement à l'information connue à la date t, sans risque de contrepartie\\
$\Pi (t, \omega) = \mathbb{E}_{t} [\pi]$ le cash-flow espéré conditionnellement à l'information connue à la date t, avec risque de contrepartie


\subsection{Pricing général}
Dans le cadre d'un marché sans possibilités d'arbitrage, le prix d'un CDS à la date t est défini par l'espérance des cash-flows futurs actualisés.

\subsubsection{Cash-flow pour un CDS sans risque}
À la date s, la firme 1 doit verser à 2 la prime $\kappa$.
 Si de plus $\tau_1 = s$, alors 1 rembourse uniquement $R_1$ de sa dette et 2 verse donc les $1-R_1$ restant à 0.
 En remarquant que le coefficient d'acutalisation entre t <  s est $\frac{\beta_{s}}{\beta_{t}}$, le cash-flow lié au CDS à la date s actualisé pour la date t est donc : 
\begin{equation*}
\frac{- \beta(s) \kappa+ \beta(\tau_{1}) \left( 1 - R_{1} \right)  \mathbf{ 1 }_{\tau_{1} } \left( s \right) }{\beta (t)} 
\end{equation*}
En intégrant entre t et $T \wedge \tau_{1}$ on trouve la formule du cash-flow futur à partir de la date t (il s'agit par exemple du prix maximal qu'un acheteur omniscient serait prêt à payer pour le contrat à la date t).

\begin{equation}
\beta (t) p(t) = -\kappa \int\limits_{t}^{T \wedge \tau_{1}} \beta(s)ds + \beta(\tau_{1})\left( 1 - R_{1} \right)\mathbf{ 1 }_{\left] t, T \right[}(\tau_{1})
\end{equation}
\subsubsection{Cas d'un CDS risqué}
On reprend la précédente analyse avec cette fois deux nouveau termes.
 Le premier correspond au défaut de 1 et 2 en même temps.
 Dans ce cas, 2 devrait rembourser $1-R_{1}$.
 Suite à son défaut, 2 ne rembourse que $R_{2}(1-R_{1})$, d'où un terme en $\mathbf{1}_{\tau_{1} = \tau_{2} = s} (1-R_{2})(1-R_{1})$.
 Par ailleurs, si seul 2 fait défaut avant 1, soit le contrat a un prix positif à cet instant et 2 doit donc a 0 $P_{s}^{+}$ mais ne lui rembourse que $R_{2}P_{s}^{+}$ car il est en défaut.
 Sinon, 0 rembourse la valeur du contrat $-P_{s}^{-}$.
 En sommant jusqu'à la fin du contrat, ie.
 $T \wedge \tau_{1} \wedge \tau_{2}$, on trouve :
\begin{multline}
\beta(t)\pi(t) = -\kappa \int\limits_{t}^{T \wedge \tau_{1} \wedge \tau_{2}} \beta(s)ds + \beta\left(\tau_{1}\right)\left( 1 - R_{1} \right)\mathbf{ 1 }_{\left] t, T \wedge \tau_{2} \right[}(\tau_{1}) \\
+ \beta(\tau_{1}) R_{2} \left( 1 - R_{1} \right) \mathbf{1}_{t < \tau_{1} = \tau_{2} < T} 
+ \beta(\tau_{2}) \left( R_{2}P_{\tau_{2}}^{+} - P_{\tau_{2}}^{-} \right) \mathbf{1}_{t < \tau_{2} < T \wedge \tau_{1}}
\end{multline}



% \subsubsection{Titre}       % Encore plus petite

% \paragraph{Titre}           % Toutes petites sections (le nom \paragraph
                              % n'est pas très bien choisi)

% \subparagraph{Titre}        % La dernière

% \appendix                   % Commençons les annexes

% \section{Titre}             % Annexe A

% \section{Titre}             % Annexe B

% \listoffigures              % Table des figures

% \listoftables               % Liste des tableaux

\end{document}


